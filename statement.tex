\documentclass[12pt]{article}
\usepackage[utf8]{inputenc}
\usepackage{amsmath}
\title{Evaluation and improvement of the accuracy of VANET simulation}

\date{\today}

\author{Andr\'e Cardote and Theodore Martin}

\begin{document}
  \maketitle 
  
\section{Problem Statement}
Vehicular Ad Hoc Networks (VANET) are known for their very specific characteristics in terms of propagation, path stability [REF] and communication time between nodes. Also, the fact that nodes are commonly cars, together with the uncontrollability of the characteristics of the vehicles that travel in roads, as well as the traffic flow,  makes it hard to deploy large scale testing and evaluations of any protocol. Moreover, reproducing any kind of scenario is almost impossible, unless there is a complete control over the vehicles that travel in a specific road. Summed up, all these premises evidence that VANET simulation is crucial for the development of protocols capable of coping with the adverse and uncontrollable conditions of such communication environment.

Due to the unique characteristics of VANETs, th
\subsection{Related Work}

\section{Objectives and Contributions of our work}

\section{Methodology}

\subsection{Analytical Evaluation}

\subsection{Real-world Experimentation}

\subsection{Simulation}


\end{document}
